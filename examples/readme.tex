% Options for packages loaded elsewhere
\PassOptionsToPackage{unicode}{hyperref}
\PassOptionsToPackage{hyphens}{url}
\PassOptionsToPackage{dvipsnames,svgnames,x11names}{xcolor}
\PassOptionsToPackage{space}{xeCJK}
%
\documentclass[
  a4paper,
]{article}
\usepackage{amsmath,amssymb}
\usepackage{iftex}
\ifPDFTeX
  \usepackage[T1]{fontenc}
  \usepackage[utf8]{inputenc}
  \usepackage{textcomp} % provide euro and other symbols
\else % if luatex or xetex
  \usepackage{unicode-math} % this also loads fontspec
  \defaultfontfeatures{Scale=MatchLowercase}
  \defaultfontfeatures[\rmfamily]{Ligatures=TeX,Scale=1}
\fi
\usepackage{lmodern}
\ifPDFTeX\else
  % xetex/luatex font selection
  \ifXeTeX
    \usepackage{xeCJK}
    \setCJKmainfont[]{Noto Serif CJK SC}
          \setCJKsansfont[]{Noto Sans CJK SC}
              \setCJKmonofont[]{Noto Sans Mono CJK SC}
      \fi
  \ifLuaTeX
    \usepackage[]{luatexja-fontspec}
    \setmainjfont[]{Noto Serif CJK SC}
  \fi
\fi
% Use upquote if available, for straight quotes in verbatim environments
\IfFileExists{upquote.sty}{\usepackage{upquote}}{}
\IfFileExists{microtype.sty}{% use microtype if available
  \usepackage[]{microtype}
  \UseMicrotypeSet[protrusion]{basicmath} % disable protrusion for tt fonts
}{}
\makeatletter
\@ifundefined{KOMAClassName}{% if non-KOMA class
  \IfFileExists{parskip.sty}{%
    \usepackage{parskip}
  }{% else
    \setlength{\parindent}{0pt}
    \setlength{\parskip}{6pt plus 2pt minus 1pt}}
}{% if KOMA class
  \KOMAoptions{parskip=half}}
\makeatother
\usepackage{xcolor}
\setlength{\emergencystretch}{3em} % prevent overfull lines
\providecommand{\tightlist}{%
  \setlength{\itemsep}{0pt}\setlength{\parskip}{0pt}}
\setcounter{secnumdepth}{-\maxdimen} % remove section numbering
\ifLuaTeX
  \usepackage{selnolig}  % disable illegal ligatures
\fi
\IfFileExists{bookmark.sty}{\usepackage{bookmark}}{\usepackage{hyperref}}
\IfFileExists{xurl.sty}{\usepackage{xurl}}{} % add URL line breaks if available
\urlstyle{same}
\hypersetup{
  pdftitle={Pandoc 模板},
  pdfauthor={David Peng},
  colorlinks=true,
  linkcolor={Maroon},
  filecolor={Maroon},
  citecolor={Blue},
  urlcolor={Blue},
  pdfcreator={LaTeX via pandoc}}

\title{Pandoc 模板}
\author{David Peng}
\date{2023-07-12}

\begin{document}
\maketitle

\hypertarget{pandoc-ux6a21ux677f}{%
\section{Pandoc 模板}\label{pandoc-ux6a21ux677f}}

自用的一些 Pandoc 模板

\hypertarget{ux4f7fux7528ux65b9ux6cd5}{%
\subsection{使用方法}\label{ux4f7fux7528ux65b9ux6cd5}}

在 Windows 上,克隆本仓库到 \texttt{\%APPDATA\%\textbackslash{}pandoc}
路径下:

\begin{verbatim}
git clone https://github.com/duzyn/pandoc-templates "%APPDATA%\pandoc"
\end{verbatim}

在 Linux 或 macOS 上,克隆本仓库到
\texttt{\textasciitilde{}/.local/share/pandoc} 路径下:

\begin{verbatim}
git clone https://github.com/duzyn/pandoc-templates ~/.local/share/pandoc
\end{verbatim}

\hypertarget{ux53c2ux8003}{%
\subsection{参考}\label{ux53c2ux8003}}

\begin{itemize}
\tightlist
\item
  \href{https://github.com/sindresorhus/github-markdown-css}{github-markdown-css}
\item
  \href{https://edwardtufte.github.io/tufte-css/}{Tufte CSS}
\item
  \href{https://rstudio.github.io/tufte/}{RStudio Tufte Handout}
\item
  \href{https://rstudio.github.io/tufte/cn/}{R Markdown Tufte Style}
\item
  \href{https://raw.githubusercontent.com/rstudio/tufte/master/inst/rmarkdown/templates/tufte_handout/resources/tufte-handout.tex}{RStudio
  Pandoc template: tufte-handout.tex}
\item
  \href{https://github.com/Wandmalfarbe/pandoc-latex-template}{Eisvogel
  LaTeX theme}
\end{itemize}

\end{document}
